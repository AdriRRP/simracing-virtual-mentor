\chapter{Introducción}
\label{cap:Introduccion}

\section{Contexto}

La tecnología de simuladores ha crecido de manera importante, llevando estas herramientas de simples representaciones a sistemas complejos que replican con alta precisión las condiciones reales de diversos entornos. En particular, los simuladores deportivos han encontrado un nicho significativo en el entrenamiento y desarrollo de habilidades en disciplinas como la conducción deportiva, conocida popularmente como Simracing. Estos simuladores permiten a los pilotos experimentar una conducción casi real, utilizando hardware especializado, como volantes y asientos, junto con software que reproduce las pistas y las leyes físicas involucradas, todo ello a un coste mucho menor comparado con el entrenamiento en pistas reales. El mundo del Simracing está dominado por varias plataformas de simulación, cada una con sus características propias, ventajes e inconvenientes. Entre las más reconocidas se encuentran iRacing \cite{iRacing}, Assetto Corsa \cite{AssetoCorsa}, RFactor \cite{rFactor} y Gran Turismo \cite{granTurismo}. iRacing, en particular,  destaca no sólo por su realismo y precisión, sino también por disponer de una comunidad activa y extensa, base de recursos para desarrollo y análisis de datos. Su realismo en la física de conducción, la variedad de vehículos y pistas, y el soporte continuo lo convierten en una de las opciones preferidas tanto por pilotos aficionados como profesionales.

\section{Análisis de la telemetría}

El análisis de telemetría es crucial para el entrenamiento en Simracing. Las plataformas como iRacing permiten extraer archivos de telemetría que capturan detalladamente los datos de la simulación, permitiendo a los pilotos y entrenadores identificar áreas de mejora. iRacing ofrece la opción de respaldar los archivos de telemetría cuando el jugador lo activa. Esta funcionalidad registra, a una frecuencia predeterminada de 60 Hz, variables como la velocidad, el porcentaje de freno, el ángulo de giro del volante, entre otros. Estas variables se registran con la frecuencia indicada, proporcionando el conjunto de datos necesario para realizar un análisis detallado y preciso del desempeño del piloto.
Para este proyecto, se ha seleccionado el formato .ibt de iRacing debido a su amplio uso y la riqueza de datos que proporciona, ideal para el análisis detallado y la extracción de métricas clave, necesarias para la mejora del rendimiento en Simracing.
Actualmente, la mayoría de las herramientas avanzadas para el análisis de estos datos son de pago, lo cual puede limitar el acceso a usuarios que no pueden permitirse estas soluciones costosas. Herramientas como Motec \cite{motec}, Atlas \cite{atlas} y Virtual Racing School \cite{vrs} ofrecen análisis profundos y detallados, pero su coste puede ser prohibitivo para muchos entusiastas del Simracing.
El análisis de telemetría implica la recolección y procesamiento de grandes cantidades de datos generados durante las sesiones de entrenamiento y competencia. Estos datos incluyen, pero no se limitan a, velocidades, fuerzas G, ángulos de dirección, uso de frenos y acelerador, entre otros. La capacidad de interpretar estos datos de manera eficaz puede marcar la diferencia entre un piloto promedio y uno de élite. Sin embargo, la interpretación requiere herramientas avanzadas que puedan procesar la información de manera rápida y precisa, además de presentarla en un formato accesible y comprensible.


\section{Motivación}

El principal motor de este proyecto es desarrollar una herramienta de análisis de telemetría de código abierto para Simracing, específicamente enfocada en la plataforma iRacing. Esta herramienta tiene como objetivo ofrecer una alternativa libre a las actuales soluciones de pago, facilitando el acceso a un análisis de telemetría avanzado a una mayor comunidad de usuarios y fomentando la colaboración comunitaria. Al ser de código abierto, permitirá a los usuarios implementar técnicas de análisis y utilizar métricas que consideren relevantes, superando las limitaciones de los productos privativos que no pueden ser extendidos fácilmente.
Además de democratizar el acceso a tecnologías avanzadas de análisis, este proyecto pretende impulsar la innovación dentro de la comunidad de Simracing. Los productos de análisis privativos suelen ser cerrados y limitados en términos de extensión, lo que puede ser una barrera para pilotos y equipos que deseen aplicar nuevas técnicas de análisis de datos o integrar métricas personalizadas.
Como parte de este proyecto, también se contempla la creación de un sistema de recomendación de acciones en lenguaje natural. Este sistema proporcionará indicaciones claras y precisas sobre las mejoras que el piloto debe realizar, basándose en el análisis detallado de la telemetría. De esta manera, se pretende no sólo facilitar el análisis de datos, sino también traducir dicho análisis en recomendaciones concretas y accionables para optimizar el rendimiento del piloto.


\section{Impacto del proyecto}

Para la comunidad de Simracing, el proyecto ofrece una herramienta poderosa y accesible que puede transformar la forma en que los pilotos analizan y mejoran su rendimiento. Al ser de código abierto, la herramienta no solo democratiza el acceso a tecnologías avanzadas de análisis, sino que también fomenta la innovación y el desarrollo continuo, permitiendo a los usuarios adaptar y mejorar la herramienta de acuerdo con sus necesidades específicas.
Además, la implementación de un sistema de recomendación de acciones en lenguaje natural puede tener un impacto significativo en la computación y en el desarrollo de sistemas recomendadores. Este enfoque no solo facilita la interpretación de los datos por parte de los pilotos, sino que también sirve como un caso de estudio para la aplicación de técnicas de agrupamiento, como K-means o \ac{fcm} en entornos dinámicos y complejos.
El proyecto también tiene el potencial de influir positivamente en la ciencia y la academia, proporcionando un recurso valioso para la investigación y la enseñanza en áreas relacionadas con la inteligencia artificial, el análisis de datos y la ingeniería de sistemas. La herramienta puede ser utilizada como un laboratorio experimental para probar y validar nuevos algoritmos y técnicas, así como para educar a futuros ingenieros en las mejores prácticas de desarrollo de software y análisis de telemetría.

\section{Estructura del documento}

Este documento presenta la siguiente estructura:

\begin{enumerate}
\item \textbf{Introducción}.
En este capítulo se establece el contexto y la motivación del proyecto, se revisa el estado actual de las herramientas de análisis de telemetría en Simracing y se describe la importancia y objetivos de desarrollar una alternativa de código abierto. Además, se proporciona una visión general de la estructura del documento, preparando al lector para los contenidos detallados que se presentarán en los capítulos siguientes.

\item \textbf{Objetivos}.
En este capítulo se describen el objetivo general y los objetivos específicos del proyecto. Se detallan las metas propuestas y se justifica la importancia de cada uno de los objetivos en el contexto del Simracing y el análisis de telemetría. Los objetivos incluyen el desarrollo de una arquitectura escalable, la implementación de un sistema eficiente de adquisición y procesamiento de datos, la creación de una interfaz de usuario intuitiva que facilite el acceso y análisis de la telemetría por parte de los pilotos, y la implementación de una interfaz en lenguaje natural para proporcionar recomendaciones de acciones a realizar por los pilotos.

\item \textbf{Antecedentes}.
En este capítulo se presentan los antecedentes teóricos y las herramientas existentes relacionadas con el proyecto. Se describen las herramientas de análisis comercial, los repositorios de interpretación de telemetrías disponibles en GitHub, y los algoritmos utilizados, como el fuzzy c-means. Este análisis proporciona el contexto necesario para comprender el estado actual de las tecnologías y metodologías en el campo del análisis de telemetría, así como las bases teóricas que sustentan el desarrollo del sistema propuesto.

\item \textbf{Plan de Gestión del Trabajo}.
Este capítulo presenta una guía de la metodología de desarrollo de software empleada, junto con otros aspectos relevantes del plan de gestión y las tecnologías utilizadas. Se discuten las razones detrás de la elección de Rust como lenguaje de desarrollo, destacando su eficiencia y seguridad en el manejo de memoria. Además, se explica la implementación de una arquitectura basada en \ac{ddd} para asegurar la escalabilidad y mantenibilidad del sistema. También se describe el cronograma del proyecto, incluyendo las principales fases y hitos, y se presentan los recursos necesarios para su implementación.

\item \textbf{Desarrollo}.
En este capítulo se detallan los requisitos del sistema, el análisis, diseño, implementación, pruebas y despliegue del asistente virtual de Simracing. Se incluye un análisis exhaustivo de los requisitos funcionales y no funcionales, así como una descripción detallada de la arquitectura del sistema. El diseño del sistema se presenta utilizando diagramas UML, incluyendo diagramas de casos de uso, diagramas de secuencia y diagramas de clases. Se discuten las decisiones de diseño clave, incluyendo la estructura de datos y los algoritmos utilizados para el procesamiento de telemetría. La implementación se describe paso a paso, destacando los desafíos encontrados y las soluciones adoptadas. Se presentan los resultados de las pruebas, incluyendo pruebas unitarias, pruebas de integración y pruebas de rendimiento, y se discute el proceso de despliegue y configuración del sistema.

\item \textbf{Conclusiones}.
Este capítulo revisa los objetivos alcanzados, las competencias adquiridas, y se discuten posibles trabajos futuros y una valoración personal del proyecto. Se evalúa el impacto del proyecto en la comunidad de Simracing y se destacan las contribuciones más significativas. Además, se proponen mejoras y extensiones futuras, incluyendo la integración con otras plataformas de simulación y la incorporación de nuevas técnicas de análisis de datos. Se incluye una reflexión personal sobre el aprendizaje y las competencias de la intensificación cursada, así como una evaluación crítica de los resultados obtenidos.

\item \textbf{Bibliografía}. Lista de las referencias bibliográficas citadas en el texto.

\item \textbf{Anexos}. Contenidos auxiliares que complementan del trabajo.
\end{enumerate}









