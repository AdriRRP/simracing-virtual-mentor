\chapter{Desarrollo}
\label{cap:Desarrollo}

En esta sección se describirá las diferentes fases del ciclo de desarrollo de software de acuerdo al plan de gestión expuesto en el capítulo \ref{cap:Planificacion}. 

\section{Requisitos y análisis del sistema}
En este apartado se presentarán los objetivos y el catálogo de requisitos del proyecto: funcionales, no funcionales, de información, reglas de negocio, etc.

Una vez catalogados los requisitos del sistema se procederá con su análisis empleando el lenguaje de modelado UML. El análisis mencionado incluirá:
\begin{itemize}
\item \textbf{Modelo conceptual}. En él se identifican clases, atributos, relacionales, etc.

\item \textbf{Modelo de casos de uso}. Representan las interacciones entre los actores y el sistema bajo estudio.

\item \textbf{Modelo de interfaz de usuario}. Puede consistir en un prototipo de baja fidelidad de la interfaz de usuario.
\end{itemize}

\subsection{Análisis de Formatos de Telemetría en Simuladores de Carreras}
Dentro del análisis del sistema, es crucial comprender los formatos de ficheros de telemetría utilizados por las plataformas de simuladores de carreras. Estos formatos presentan particularidades que afectan la extracción y gestión de datos. A continuación, se detallan algunos aspectos relevantes:

\begin{itemize}
\item \textbf{Formatos Propietarios y Herramientas de Extracción}. La mayoría de los simuladores de carreras utilizan formatos binarios propietarios para sus ficheros de telemetría, cuya especificación no es de dominio público. Esto implica dificultades para acceder y comprender estos datos, ya que se requiere de herramientas específicas para su extracción y análisis.

\item \textbf{Elección de iRacing y su Formato .ibt}. iRacing destaca como una plataforma de sim-racing ampliamente utilizada, lo que ha motivado el desarrollo de herramientas libres para la gestión de sus ficheros de telemetría en formato .ibt. Este formato binario contiene información detallada sobre parámetros como posición en pista, velocidad, aceleración, frenado, entradas del piloto, entre otros.
\item \textbf{Herramientas de Extracción y Análisis}. Dentro de la comunidad de sim-racing, existen diversas herramientas de código abierto que permiten la extracción y análisis de los ficheros .ibt de iRacing. Estas herramientas facilitan la visualización de datos de rendimiento, la comparación de sesiones y la mejora del desempeño en pista.
\end{itemize}

\section{Diseño del sistema}
En esta sección se define la arquitectura lógica general del sistema. Para describirla se puede emplear un modelo como C4~\cite{Brown22}, el cual permite representar la arquitectura de un sistema software mediante varios diagramas a distintos niveles de abstracción.

\section{Implementación del sistema}
En este apartado se describirá la organización del código fuente y \emph{scripts}, describiendo el propósito de los distintos ficheros y su distribución en paquetes y directorios. Puede ser conveniente incluir alguna porción significativa de código fuente que sea de especial relevancia por su funcionalidad.

En el desarrollo de proyectos software cobra especial importancia el empleo de sistemas de control de versiones junto a repositorios en línea. Estas herramientas se convierten en esenciales para disponer de un registro histórico del desarrollo que también puede ayudar a evaluar el trabajo realizado. Por este motivo, en la memoria del proyecto se debe indicar la dirección de los repositorios empleados (p.~ej.,~Github).

\section{Pruebas del sistema}
En esta sección se describirá el plan de pruebas del sistema incluyendo todos los tipos llevados a cabo. El desarrollo de la sección debería tratar los aspectos siguientes:
\begin{itemize}[noitemsep]
\item \textbf{Estrategia}. Donde se indica el alcance de las pruebas y los procedimientos.

\item \textbf{Pruebas unitarias}. Destinadas a localizar errores en cada nuevo módulo software desarrollado antes de su integración con el resto del sistema.

\item \textbf{Pruebas de integración}. Su objetivo es localizar errores en subsistemas completos analizando la interacción entre varios módulos de software.

\item \textbf{Pruebas de sistema}. Contempla las pruebas funcionales con las que se realiza el análisis del buen funcionamiento de la implementación de los casos de uso del sistema. Además, en estas pruebas se comprueba el funcionamiento respecto a los requisitos no funcionales: eficiencia, seguridad, etc.

\item \textbf{Pruebas de aceptación}. Su intención es demostrar, con la participación del cliente, que el producto está listo para su puesta en funcionamiento en el entorno producción.
\end{itemize}

\section{Despliegue}
Esta sección recoge la arquitectura física propuesta del sistema, las instrucciones para su despliegue, operación y mantenimiento del
nivel de servicio.

Es muy importante que todas las justificaciones aportadas se sustenten no solo en juicios de valor sino en evidencias tangibles como: historiales de actividad, repositorios de código y documentación, porciones de código, trazas de ejecución, capturas de pantalla, demos, etc.
