\chapter{Conclusiones}
\label{cap:Conclusiones}




\section{Objetivos alcanzados}
En este capítulo se realizará un juicio crítico y discusión sobre el objetivo general y objetivos secundarios alcanzados. 





\section{Justificación de competencias adquiridas}
Es muy importante recordar que según la normativa vigente en la ESI-UCLM, el capítulo de conclusiones debe incluir \emph{obligatoriamente} un apartado destinado a justificar la aplicación en el TFG de las competencias específicas (una o más) adquiridas en la tecnología específica cursada, como se indica a continuación:

En el TFG se han aplicado las competencias correspondientes a la Tecnología Específica de \emph{[poner lo que corresponda]}:

\begin{description}
\item[Código de la competencia 1:] \emph{[Texto de la competencia 1]}. Explicación de cómo se ha aplicado en el TFG.
\item[\dots] (otras más si las hubiera).
\end{description}




\section{Trabajos derivados y futuros}
Si es pertinente se puede incluir información sobre trabajos derivados como publicaciones o ponencias en preparación, así como trabajos futuros \emph{(solo si estos están iniciados o planificados en el momento que se redacta el texto)}.

Se recomienda evitar la inclusión de una lista de posibles mejoras. Contrariamente a lo que se pueda pensar, pueden transmitir la impresión de que el trabajo se encuentra en un estado incompleto o inacabado.\footnote{En cualquier caso se debe reflexionar sobre este aspecto por si los miembros del comité evaluador preguntan sobre estas posibles mejoras durante la defensa del trabajo.}




\section{Valoración personal}
En esta sección final se realizará un rápido análisis de las lecciones aprendidas en las que se pueden incluir tanto buenas prácticas adquiridas (tecnológicas y procedimentales) como cualquier otro aspecto de interés. También se puede resumir cuantitativamente el tiempo y esfuerzo dedicados al proyecto a lo largo de su desarrollo.







