\chapter{Objetivo}
\label{cap:Objetivo}
Este capítulo define los objetivos del trabajo, especificando tanto el objetivo general como los objetivos específicos.


\section{Objetivo General}

El objetivo principal de este \ac{tfg} es diseñar y desarrollar un entrenador virtual de código abierto capaz de interpretar archivos de telemetría del software iRacing. Este entrenador virtual proporcionará un análisis detallado, permitiendo la visualización de métricas comparativas entre dos vueltas en un circuito, y ofrecerá recomendaciones de mejora de la conducción en lenguaje natural. De este modo, facilitará la comprensión y aplicación de las mejoras por parte de los pilotos, contribuyendo a su rendimiento y desarrollo en el Simracing.

\section{Objetivos Específicos}
Para alcanzar el objetivo general, se plantean los siguientes objetivos específicos:

\subsection{Adquirir y procesar datos de telemetría}
\begin{itemize}
    \item Investigar cómo se adquieren y transfieren los ficheros de telemetría .ibt de iRacing y conseguir una muestra.
    \item Estudiar el formato de telemetría .ibt de iRacing.
    \item Desarrollo de un módulo para la lectura y decodificación de archivos de telemetría en formato .ibt.
\end{itemize}

\subsection{Fijar las métricas de comparación}
\begin{itemize}
    \item Identificar y seleccionar las variables de telemetría relevantes para su posterior comparación.
    \item Desarrollar un modelo que defina y permita comparar los datos de telemetría.
\end{itemize}

\subsection{Comparar datos de telemetría y aplicar aprendizaje no supervisado para clasificar diferencias}
\begin{itemize}
    \item Implementar métodos para la comparación de datos de telemetría.
    \item Aplicar técnicas de aprendizaje no supervisado para clasificar las diferencias observadas.
\end{itemize}

\subsection{Construir el conjunto de sugerencias con una respuesta en lenguaje natural}
\begin{itemize}
    \item Desarrollar un método sistemático para la asignación de etiquetas interpretables a las diferencias en los datos de telemetría.
    \item Crear un sistema de generación de sugerencias en lenguaje natural basado en la interpretación de las diferencias.
    \item Diseñar una interfaz que presente las sugerencias de manera clara y comprensible para el usuario.
\end{itemize}

\subsection{Desarrollar una aplicación web donde se realicen todas las acciones y se muestre gráficamente los resultados}
\begin{itemize}
    \item Investigar y evaluar lenguajes y frameworks adecuados para el desarrollo de aplicaciones web con gráficos interactivos y algoritmos complejos.
    \item Analizar los requisitos de rendimiento y escalabilidad necesarios para gestionar eficientemente grandes volúmenes de datos.
    \item Diseñar una arquitectura que soporte la escalabilidad y la modularidad para facilitar el desarrollo y mantenimiento.
    \item Explorar opciones de bases de datos que proporcionen simplicidad y flexibilidad en los cambios de esquema.
    \item Desarrollar una interfaz de usuario intuitiva y eficiente que permita la visualización gráfica de los resultados.
    \item Asegurar que la aplicación sea extensible y adaptable para futuras mejoras y colaboraciones en la comunidad de código abierto.
    
\end{itemize}



\section{Requisitos de la Solución}
La solución deberá cumplir con los siguientes requisitos:

\subsection{Requisitos Funcionales}
\begin{itemize}
    \item Permitir la carga y procesamiento de archivos de telemetría en formato .ibt de iRacing.
    \item Proporcionar visualizaciones gráficas claras de las métricas comparativas.
    \item Ofrecer análisis y recomendaciones basadas en diferencias de rendimiento entre dos vueltas.
\end{itemize}

\subsection{Requisitos No Funcionales}
\begin{itemize}
    \item Ser accesible desde diferentes dispositivos y navegadores.
    \item Facilitar la colaboración y escalabilidad en el desarrollo, permitiendo la ampliación de la aplicación por múltiples desarrolladores.
\end{itemize}