%--- Ajustes del documento.
\pagestyle{plain}	% Páginas sólo con numeración inferior al pie

% -------------------------
%
% RESUMEN:
% 
%

% EDITAR: Resumen (máx. 1 pág.)
%\cleardoublepage % Se incluye para modificar el contador de página antes de añadir bookmark
\phantomsection  % Necesario con hyperref
\addcontentsline{toc}{chapter}{Resumen} % Añade al TOC.
\selectlanguage{spanish} % Selección de idioma del resumen.
\makeatletter
\begin{center} %
   {\textsc{TRABAJO FIN DE GRADO - ESCUELA SUPERIOR DE INFORMÁTICA 
   (UCLM)}\par} % Tipo de trabajo
   \vspace{1cm} %  
   {\textbf{\Large\@tituloCorto}\par}  % Título del trabajo
   \vspace{0.4cm} %
   {\@autor \\ \@cityTF,{} \@mesTF{} \@yearTF\par} 
   \vspace{0.9cm} %
   {\textbf{\large\textsf{Resumen}}\par} % Título de resumen
\end{center}   
\makeatother %

Los simuladores han experimentado una evolución significativa, pasando de ser modelos simplificados de la realidad a sistemas complejos que reflejan con precisión los escenarios que simulan. Esta capacidad de imitar casi a la perfección los sistemas simulados convierte a los simuladores en herramientas eficaces para el aprendizaje y el entrenamiento. Un campo cuya popularidad  ha crecido de manera notable es el de la conducción deportiva, también conocida como Simracing. En estos sistemas, el piloto utiliza hardware y software con capacidad de simular el vehículo, la pista y las leyes físicas que rigen las interacciones entre los diferentes elementos, permitiendo así realizar sesiones de entrenamiento a un coste económico reducido.
El proceso de entrenamiento y aprendizaje suele ser autónomo, de tal manera que el piloto puede observar las diferentes telemetrías que proporciona el simulador e intenta mejorarlas. Sin embargo, el aprendizaje puede estar asistido por un entrenador, ya sea virtual o humano, que aporta el conocimiento y la experiencia necesarios para indicar las mejoras a realizar.
En este trabajo se propone la definición y construcción de un entrenador virtual de Simracing, proporcionando una alternativa de código abierto a las herramientas de pago disponibles en el mercado. El funcionamiento de este entrenador virtual se basará en el estudio de la telemetría del jugador, a partir de la cual se obtendrá un conjunto de datos que se compararán con valores de referencia. Para realizar dicha comparación se utilizarán técnicas de aprendizaje no supervisado, concretamente el agrupamiento difuso, para asociar las diferencias y además proporcionar indicaciones en lenguaje natural sobre las acciones que requieren mejora.
Se pretende que este proyecto crezca en el futuro, extendiendo las características del asistente virtual y promoviendo así una herramienta accesible y colaborativa para la mejora del rendimiento en juegos tipo Simracing.


%En una página como máximo, el resumen explica de modo conciso la problemática que trata de resolver el trabajo \emph{(<<Qué>>)}, la metodología para  abordar su solución (\emph{<<Cómo>>)} y las principales conclusiones del trabajo. En los trabajos cuyo idioma principal sea el inglés, el orden de \textsf{Resumen} y \textsf{Abstract} se invertirá.

%En concreto este documento debe servir como guía para preparar, con \LaTeX, el TFG en la \href{http://webpub.esi.uclm.es/}{Escuela Superior de Informática} (ESI) de la Univ. de Castilla-La Mancha (UCLM), siguiendo la \href{https://pruebasaluuclm.sharepoint.com/sites/esicr/tfg/SitePages/Inicio.aspx}{normativa de aplicación}. Está disponible tanto en \href{https://github.com/JesusSalido/TFG_ESI_UCLM}{GitHub} como \href{https://www.overleaf.com/latex/templates/plantilla-de-tfg-escuela-superior-de-informatica-uclm/phjgscmfqtsw}{Overleaf}. Por tanto, se puede emplear, tanto en un equipo con \LaTeX{} (modo local) instalado, o bien empleando el servicio de edición \href{https://www.overleaf.com/latex/templates/plantilla-de-tfg-escuela-superior-de-informatica-uclm/phjgscmfqtsw}{Overleaf} (modo online).

%Este texto se aprovecha para proporcionar información sobre la elaboración de la memoria del TFG con ayuda de \LaTeX{} empleando este documento como plantilla. Por este motivo, sigue una estructura similar a la que se espera encontrar en un TFG, mostrando ejemplos de uso de distintos elementos y comandos de maquetación de textos que se amplía en el anexo~\ref{cap:AnexoA}.

%\noindent\emph{IMPORTANTE: Aunque la plantilla se ajusta a las necesidades y reglamentación de la ESI-UCLM, se puede adaptar fácilmente a otras titulaciones, instituciones y otros documentos de carácter académico. Esta plantilla permite la elaboración automática del documento en idioma inglés en cualquier SO (Windows, Linux, Mac OSX, etc.).}

\bigskip
\noindent\textbf{Palabras clave}: Aprendizaje Automático, Entrenador Virtual, Simulación, Telemetría, Simracing %\emph{(como mucho 5 palabras que se puedan emplear como etiquetas de búsqueda)}.

%---
\cleardoublepage % Se incluye para modificar el contador de página antes de añadir 





% EDITAR: Abstract (máx. 1 pág.)
%---
\phantomsection  % Necesario con hyperref
\addcontentsline{toc}{chapter}{Abstract} % Añade al TOC.
\selectlanguage{english} % Selección de idioma del resumen.
\makeatletter
\begin{center} %
   {\textsc{BACHELOR DISSERTATION - ESCUELA SUPERIOR DE INFORMÁTICA 
   (UCLM)}\par}
   \vspace{1cm} %  
   {\textbf{\Large Virtual Assistant for Sports Driving Simulation}\par}
   \vspace{0.4cm} %
   {\@autor \\ \@cityTF,{} \@monthTF{} \@yearTF\par} 
   \vspace{0.9cm} %
   {\textbf{\large\textsf{Abstract}}\par} 
\end{center}   
\makeatother %

Simulators have undergone significant evolution, transitioning from simplified models of reality to complex systems that accurately reflect the scenarios they simulate. This ability to almost perfectly mimic the simulated systems makes simulators effective tools for learning and training. One field whose popularity has grown notably is that of sports driving, also known as Simracing. In these systems, the driver uses hardware and software capable of simulating the vehicle, the track, and the physical laws governing the interactions between different elements, thereby enabling training sessions at a reduced economic cost.
The training and learning process is usually autonomous, allowing the driver to observe the various telemetries provided by the simulator and attempt to improve them. However, learning can also be assisted by a coach, either virtual or human, who brings the necessary knowledge and experience to indicate the improvements to be made.
This paper proposes the definition and construction of a virtual Simracing coach, providing an open-source alternative to the paid tools available on the market. The operation of this virtual coach will be based on the study of the player's telemetry, from which a dataset will be obtained and compared with reference values. To perform this comparison, unsupervised learning techniques will be used, specifically fuzzy clustering, to associate the differences and additionally provide natural language indications on the actions that require improvement.
This project is intended to grow in the future, extending the characteristics of the virtual assistant and thus promoting an accessible and collaborative tool for improving performance in Simracing games.

\bigskip 

\noindent\textbf{Keywords}: Machine Learning, Virtual Coach, Simulation, Telemetry, Simracing

\cleardoublepage % Se incluye para modificar el contador de página antes de añadir 

