% -------------------------
%
% -NOTACIÓN: Lista de símbolos con significado especial.
%
% -------------------------
\cleardoublepage
\phantomsection % Necesario con hyperref

% El método mostrado en este fichero es un modo rápido de incluir nomeclatura y listade acrónimos. En trabajos donde se precise un trabajo más depurado e intensivo puede recurrirse a los paquetes:
%   - nomencl
%   - acronym

\chapter*{Notación y acrónimos} % Opción con * para que no aparezca en TOC ni numerada
\addcontentsline{toc}{chapter}{Notación y acrónimos} % Añade al TOC.

\section*{Notacion}
% Ejemplo de lista con notación (o nomenclatura) empleada en la memoria del TFG.\footnote{Se incluye únicamente con propósito de ilustración, ya que el documento no emplea la notación aquí mostrada.}

\begin{tabular}{r r p{0.8\linewidth}}
% $A, B, C, D$	& : & Variables lógicas \\
% $f, g, h$		& :	& Funciones lógicas \\
% $\cdot$			& : & Producto lógico (AND), a menudo se omitirá como en $A 
% B$ en lugar de $A \cdot B$\\
% $+$				& : & Suma aritmética o lógica (OR) dependiendo del 
% contexto\\
% $\oplus$		& : & OR exclusivo (XOR)\\
% $\overline{A}$ o ${A}'$	& : & Operador NOT o negación
\end{tabular}

\section*{Lista de acrónimos}
% OJO: Esta lista debería estar ordenada alfabeticamente (hacer de modo manual).
% Ejemplo de lista \emph{ordenada alfabéticamente} con los acrónimos empleados en el texto.\footnote{Se pueden omitir aquellos acrónimos que son reconocidos en el contexto académico (p.~ej., PhD), aunque aquí se han incluido a efectos ilustrativos.}

\begin{acronym}[ICANN]
    \acro  {acti} [ACTI] {Assetto Corsa Telemetry Interface}
    \acro  {api} [API] {Application Programming Interface}
    \acro  {cicd} [CI/CD] {Continuous Integration and Continuous Deployment/Delivery}
    \acro  {css} [CSS] {Cascading Style Sheets}
    \acro  {csv} [CSV] {Comma-Separated Values}
    \acro  {ddd} [DDD] {Domain Driven Design}
    \acro  {dom} [DOM] {Document Object Model}
    \acro  {fcm} [FCM] {Fuzzy C-means}
    \acro  {fpc} [FPC] {Fuzzy Partition Coefficient}
    \acro  {html} [HTML] {HyperText Markup Language}
    \acro  {http} [HTTP] {HyperText Transfer Protocol}
    \acro  {ibt}   [IBT]   {iRacing Binary Telemetry}
    \acro  {ide}   [IDE]   {Integrated Development Environment}
    \acro  {json}   [JSON]   {JavaScript Object Notation}
    \acro  {pln} [PLN] {Procesamiento de Lenguaje Natural}
    \acro  {rest} [REST] {Representational State Transfer}
    \acro  {sdk}   [SDK]   {Software Development Kit}
    \acro  {tfg}   [TFG]   {Trabajo Fin de Grado}
    \acro  {toml}   [TOML]   {Tom's Obvious Minimal Language}
    \acro  {xp}   [XP]   {eXtreme Programming}
    \acro  {vrs}   [VRS]   {Virtual Racing School}
    \acro  {uuid}   [UUID]   {Universally Unique Identifier}
    \acro  {wasm}   [WASM]   {WebAssembly}
    \acro  {yaml}   [YAML]   {YAML Ain't Markup Language}
\end{acronym}

% \begin{tabular}{r r p{0.8\linewidth}}
% CASE& : &Computer-Aided Software Engineering \\
% CTAN& : &Comprenhensive \TeX{} Archive network \\
% IDE& : &Integrated Development Environment \\
% DDD& : &Domain Driven Design \\
% ECTS& : &European Credit Transfer and Accumulation System \\
% OOD& : &Object-Oriented Design \\
% PhD& : &Philosophiae Doctor \\
% RAD& : &Rapid Application Development \\
% SDLC& : &Software Development Life Cycle \\
% SSADM& : &Structured Systems Analysis \& Design Method \\
% TFE& : &Trabajo Fin de Estudios \\
% TFG& : &Trabajo Fin de Grado \\
% TFM& : &Trabajo Fin de Máster \\
% UML& : &Unified Modeling Language
% \end{tabular}
